\documentclass[12pt]{article}

\usepackage[T2A]{fontenc}
\usepackage{fullpage}
\usepackage{multicol,multirow}
\usepackage{tabularx}
\usepackage{ulem}
\usepackage[utf8]{inputenc}
\usepackage[english, russian]{babel}
\usepackage{amsmath}
\usepackage{amssymb}
\usepackage{graphicx}

\usepackage{titlesec}

\usepackage{erewhon}

\renewcommand{\rmdefault}{erewhon}
\titleformat{\section}
  {\normalfont\Large\bfseries}{\thesection.}{0.3em}{}

\titleformat{\subsection}
  {\normalfont\large\bfseries}{\thesubsection.}{0.3em}{}

\titlespacing{\section}{0pt}{*2}{*2}
\titlespacing{\subsection}{0pt}{*1}{*1}
\titlespacing{\subsubsection}{0pt}{*0}{*0}
\usepackage{listings}
\lstloadlanguages{Lisp}
\lstset{extendedchars=false,
	breaklines=true,
	breakatwhitespace=true,
	keepspaces = true,
	tabsize=2
}
\begin{document}


\section*{Отчет по лабораторной работе №\,4 
по курсу \\
\guillemotleft Функциональное программирование\guillemotright}
\begin{flushright}
Студент группы 8О-306 МАИ \textit{Наумов Дмитрий}, \textnumero 15 по списку \\
\makebox[7cm]{Контакты: {\tt dandachok@gmail.com} \hfill} \\
\makebox[7cm]{Работа выполнена: 27.04.2020 \hfill} \\
\ \\
Преподаватель: Иванов Дмитрий Анатольевич, доц. каф. 806 \\
\makebox[7cm]{Отчет сдан: \hfill} \\
\makebox[7cm]{Итоговая оценка: \hfill} \\
\makebox[7cm]{Подпись преподавателя: \hfill} \\

\end{flushright}

\section{Тема работы}
Знаки и строки

\section{Цель работы}
Научиться работать с литерами (знаками) и строками при помощи функций обработки строк и общих функций работы с последовательностями.

\section{Задание (вариант №43)}
Запрограммировать на языке Коммон Лисп функцию, принимающую один аргумент - текст.

Функция должна удалить все слова, в которых встречается не более двух различных букв, и вернуть новый текст. Сравнение как латинских букв, так и русских должно быть регистро-независимым.
\begin{lstlisting}[language=lisp]
(remove-two-char-words '("Оно скрылось за деревьями." 
                        "Мы прошли, не заметив его."))
 => ("скрылось деревьями." "прошли, заметив его.")
\end{lstlisting}
\section{Оборудование студента}
Процессор Intel Core i3-2100 4\,@\,2.2GHz, память: 8Gb, разрядность системы: 64.

\section{Программное обеспечение}
ОС Ubuntu 18.04.4 LTS, clisp 2.49.60

\pagebreak

\section{Идея, метод, алгоритм}
Функция {\tt russian-upper-case-p} проверяет, находится ли буква в верхнем регистре, если да, то возвращает ее номер.

{\tt russian-char-downcase} возвращает символ в нижнем регистре.

{\tt russian-string-downcase} возращает строку в нижнем регистре.

Функция {\tt more-two-symbol-p} возвращает {\tt T}, если строка содержит больше двух букв, {\tt NIL} в противном случае.
Реализация довольно простая, помещаем каждую букву в список, если ее там еще нет. Затем проверяем размер списка.

Функция {\tt remove-two-char-words} принимает список строк. В каждой строке выделяем слова разделенные пробелом и проверяем их на количество букв, если букв больше двух, то добавляем слово в результирующую строку.

\section{Распечатка программы и её результаты}
\subsection{Исходный код}
\begin{lstlisting} [language=lisp]
(defun russian-upper-case-p (char)
  (position char "АБВГДЕЁЖЗИЙКЛМНОПРСТУФХЦЧШЩЪЫЬЭЮЯ"))

(defun russian-char-downcase (char)
  (let ((i (russian-upper-case-p char)))
    (if i 
        (char "абвгдеёжзийклмнопрстуфхцчшщъыьэюя" i)
        (char-downcase char)))) 

(defun russian-string-downcase (string)
  (map 'string #'russian-char-downcase string))

(defun more-two-symbol-p (string)
  (let ((str (string-downcase (string-right-trim ",.:!?" string))) (symbols (list)))
    (let ((len (length str)))
    (do ((i 0 (+ i 1)))
      ((>= i len))
      (if (not (member (char str i) symbols))
        (setf symbols (append symbols (list (char str i))))))
    (> (length symbols) 2))))

(defun space-p (char)
  (member char '(#\Space #\Tab #\Newline)))

(defun remove-two-char-words (text)
  (let (ans (list))
    (dolist (sentence text)
      (let ((len (length sentence))
            (right 0) 
            (ans-str "")
            (word "")
            (more-two NIL))
        (do ((left 0 (+ left 1)))
            ((>= left len))
            (setf right (or (position-if #'space-p sentence :start left) len))
            (setf word (subseq sentence left right))
            (setf more-two (more-two-symbol-p word))
            (if more-two
              (setf ans-str (concatenate 'string ans-str word)))
            (if (and (< right len) (= right left))
              (setf ans-str (concatenate 'string ans-str 
                (make-string 1 :initial-element (char sentence right)))))
            (setf left right))
        (setf ans (append ans (list ans-str)))))
      ans))
\end{lstlisting}

\subsection{Результаты работы}
\begin{lstlisting}
dandachok@dpc:~/Documents/Study/sem6/FP/lab4$ clisp
[1]> (load "solution.lisp")
;; Loading file solution.lisp ...
;; Loaded file solution.lisp
#P"/home/dandachok/Documents/Study/sem6/FP/lab4/solution.lisp"
[2]> (remove-two-char-words '(" Оно скрылось за деревьями." " Мы прошли, не заметив его."))
(" скрылось деревьями." " прошли, заметив его.")
[7]> (remove-two-char-words '(" Оно она онн"))
(" она ")
[7]> (remove-two-char-words '(" Оно она онн. ja jj jaa adskf."))
(" она adskf.")
[7]> (remove-two-char-words '(" Оно она онн. ja jJ jaA adskf."))
(" она adskf.") 
\end{lstlisting}

\section{Дневник отладки}
\begin{tabular}{|c|p{5cm}|p{5cm}|c|}
\hline
Дата & Событие & Действие по исправлению & Примечание \\
\hline
06.05.2020 & Некорректная работа с кириллицей & Добавлены функции для работы с кириллицей & \\
\hline
\end{tabular}

\section{Замечания автора по существу работы}

\begin{lstlisting}[language=lisp]

\end{lstlisting}

\section{Выводы}
Сложность программы по времени и по памяти линейная.
\end{document}