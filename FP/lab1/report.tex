\documentclass[12pt]{article}

\usepackage{fullpage}
\usepackage{multicol,multirow}
\usepackage{tabularx}
\usepackage{ulem}
\usepackage[utf8]{inputenc}
\usepackage[russian]{babel}
\usepackage{amsmath}
\usepackage{amssymb}

\usepackage{titlesec}

\titleformat{\section}
  {\normalfont\Large\bfseries}{\thesection.}{0.3em}{}

\titleformat{\subsection}
  {\normalfont\large\bfseries}{\thesubsection.}{0.3em}{}

\titlespacing{\section}{0pt}{*2}{*2}
\titlespacing{\subsection}{0pt}{*1}{*1}
\titlespacing{\subsubsection}{0pt}{*0}{*0}
\usepackage{listings}
\lstloadlanguages{Lisp}
\lstset{extendedchars=false,
	breaklines=true,
	breakatwhitespace=true,
	keepspaces = true,
	tabsize=2
}
\begin{document}


\section*{Отчет по лабораторной работе №\,2 
по курсу \guillemotleft  Функциональное программирование\guillemotright}
\begin{flushright}
Студент группы 8О-306 МАИ \textit{Наумов Дмитрий}, \textnumero 15 по списку \\
\makebox[7cm]{Контакты: {\tt dandachok@gmail.com} \hfill} \\
\makebox[7cm]{Работа выполнена: 16.03.2020 \hfill} \\
\ \\
Преподаватель: Иванов Дмитрий Анатольевич, доц. каф. 806 \\
\makebox[7cm]{Отчет сдан: \hfill} \\
\makebox[7cm]{Итоговая оценка: \hfill} \\
\makebox[7cm]{Подпись преподавателя: \hfill} \\

\end{flushright}

\section{Тема работы}
Примитивные функции и особые операторы Коммон Лисп

\section{Цель работы}
Научиться вводить S-выражения в Лисп-систему, определять переменные и функции, работать с условными операторами, работать с числами, использую схему линейной и древовидной рекурсии.

\section{Задание (вариант №48)}
Реализуйте на языке Коммон Лисп программу для умножения двух целых чисел за логарифмическое число шагов. Можно использовать функции сложения, вычитания, умножения и деления числа на 2, но нельзя умножать произвольные числа.

\section{Оборудование студента}
Процессор Intel Core i5-5200 4\,@\,2.2GHz, память: 4Gb, разрядность системы: 64.

\section{Программное обеспечение}
ОС Ubuntu 18.04.4 LTS, clisp 2.49.60

\section{Идея, метод, алгоритм}
Функция {\tt fast*} рекурсивна и работает следующим образом:
\begin{itemize}
\setlength{\itemsep}{-1mm} % уменьшает расстояние между элементами списка
\item если один из аргументов ноль, то вернет ноль, иначе
\item если второй аргумент равен {\tt 1} или {\tt -1}, то вернет {\tt a} и {\tt -a} соответственно, иначе
\item если второй аргумент меньше нуля, то вернет значение функции(как если бы второй аргумент был положительным) с противоположным знаком, иначе
\item если {\tt b} нечетное, то вернет  сумму {\tt a} и разультат умножения {\tt a} на {\tt b-1}, иначе
\item вернет результат умножения {\tt a} на {\tt b/2} умноженный на 2 
\end{itemize}

\section{Распечатка программы и её результаты}
\subsection{Исходный код}
\begin{lstlisting} [language=lisp]
(defun fast* (a b)
    (if (or (= a 0) (= b 0))
        0
        (if (= b 1)
            a
            (if (= b (- 1))
                (- a)
                (if (< b 0)
                    (- (fast* a (- b)))
                    (if (= (mod b 2) 1)
                        (+ a (fast* a (- b 1)))
                        (* 2 (fast* a (/ b 2)))))))))
\end{lstlisting}

\subsection{Результаты работы}
\subsection{Исходный код}

\subsection{Результаты работы}
\begin{lstlisting} [language=terminal]
eldar@eldar-Aspire-E5-573G:~/fp/dima/task1$ clisp
[1]> (load "./solution")
;; Загружается файл /home/eldar/fp/dima/task1/solution.fas...
;; Загружен файл /home/eldar/fp/dima/task1/solution.fas
#P"/home/eldar/fp/dima/task1/solution.fas"
[2]> (fast* 1 0)
0
[3]> (fast* 0 1)
0
[4]> (fast* 1 1)
1
[5]> (fast* 2 -1)
-2
[6]> (fast* 2 5)
10
[7]> (fast* 2 4)
8
[8]> (fast* -2 4)
-8
[9]> (fast* -2 -4)
8
[10]> (fast* 2 -4)
-8
[11]> (fast* 123 345)
42435
[12]> (exit)
\end{lstlisting}

\section{Дневник отладки}
\begin{tabular}{|c|c|c|c|}
\hline
Дата & Событие & Действие по исправлению & Примечание \\
\hline
\end{tabular}

\section{Замечания автора по существу работы}
Данное задание схоже с быстрым возведением в степень, которое мне приходилось реализовывать на 2 курсе, поэтому с алгоритмом проблем не возникло.

\section{Выводы}
Программа работает за логарифмическое время. Но смысла в этом мало, т.к. аппаратно умножение производиться за константу, но реализованное аналогично возведение в степень будет быстрее, чем последовательное умножение.

\end{document}
