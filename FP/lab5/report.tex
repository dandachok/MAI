\documentclass[12pt]{article}

\usepackage[T2A]{fontenc}
\usepackage{fullpage}
\usepackage{multicol,multirow}
\usepackage{tabularx}
\usepackage{ulem}
\usepackage[utf8]{inputenc}
\usepackage[english, russian]{babel}
\usepackage{amsmath}
\usepackage{amssymb}
\usepackage{graphicx}

\usepackage{titlesec}

\usepackage{erewhon}

\renewcommand{\rmdefault}{erewhon}
\titleformat{\section}
  {\normalfont\Large\bfseries}{\thesection.}{0.3em}{}

\titleformat{\subsection}
  {\normalfont\large\bfseries}{\thesubsection.}{0.3em}{}

\titlespacing{\section}{0pt}{*2}{*2}
\titlespacing{\subsection}{0pt}{*1}{*1}
\titlespacing{\subsubsection}{0pt}{*0}{*0}
\usepackage{listings}
\lstloadlanguages{Lisp}
\lstset{extendedchars=false,
	breaklines=true,
	breakatwhitespace=true,
	keepspaces = true,
	tabsize=2
}
\begin{document}


\section*{Отчет по лабораторной работе №\,5 по курсу \\
\guillemotleft Функциональное программирование\guillemotright}
\begin{flushright}
Студент группы 8О-306 МАИ \textit{Наумов Дмитрий}, \textnumero 15 по списку \\
\makebox[7cm]{Контакты: {\tt dandachok@gmail.com} \hfill} \\
\makebox[7cm]{Работа выполнена: 27.04.2020 \hfill} \\
\ \\
Преподаватель: Иванов Дмитрий Анатольевич, доц. каф. 806 \\
\makebox[7cm]{Отчет сдан: \hfill} \\
\makebox[7cm]{Итоговая оценка: \hfill} \\
\makebox[7cm]{Подпись преподавателя: \hfill} \\

\end{flushright}

\section{Тема работы}
Обобщённые функции, методы и классы объектов

\section{Цель работы}
Научиться определять простейшие классы, порождать экземпляры классов, считывать и изменять значения слотов, научиться определять обобщённые функции и методы.

\section{Задание (вариант №41)}
Определить обычную функцию {\tt on-single-line-p} - предикат,

\begin{itemize}
  \item принимающий в качестве аргумента список точек (радиус-векторов),
  \item возвращающий {\tt T}, если все указанные точки лежат на одной прямой (вычислять с допустимым отклонением {\tt *tolerance*}).
\end{itemize}

Точки могут быть заданы как декартовыми координатами (экземплярами {\tt cart}), так и полярными (экземплярами {\tt polar}).

\begin{lstlisting}[language=lisp]
  (defvar *tolerance* 0.001)

  (defun on-single-line-p (vertices)
    ...)
\end{lstlisting}
\section{Оборудование студента}
Процессор Intel Core i3-2100 4\,@\,2.2GHz, память: 8Gb, разрядность системы: 64.

\section{Программное обеспечение}
ОС Ubuntu 18.04.4 LTS, clisp 2.49.60

\section{Идея, метод, алгоритм}
Программа состоит из следующих функций:
\begin{itemize}
  \item {\tt cart-x} вычисляет $x$-координату из полярных
  \item {\tt cart-y} вычисляет $y$-координату из полярных
  \item {\tt to-cart} преобразует координаты в декартовое представление:
  
    \begin{itemize}
      \item если передается объект {\tt cart}, то возвращает его
      \item если передается объект {\tt polar}, то вычисляет $x, y$ и создает объект класса {\tt cart}
    \end{itemize}
  \item {\tt on-line-p} проверяет, находится ли точка на прямой в каноническом представлении с параметрами $x1, y1, x2, y2$
    
    Функция просто проверяет выполнение следующего условия:
    \[\frac{x - x_1}{x_1 - x_2} - \frac{y - y_1}{y_1 - y_2} \approx 0 \]
  \item {\tt on-single-line-p} проверяет все точки списка на пренадлежность прямой
    
    В качестве параметров прямой берутся координаты первых двух точек.
\end{itemize}

\pagebreak

\section{Распечатка программы и её результаты}
\subsection{Исходный код}
\begin{lstlisting} [language=lisp]
  (defvar *tolerance* 0.001)
  (defclass cart ()
   ((x :initarg :x :reader x)
    (y :initarg :y :reader y)))
  (defclass polar ()
   ((radius :initarg :radius :accessor radius)
    (angle  :initarg :angle  :accessor angle)))
  (defmethod cart-x ((p polar))
    (* (radius p) (cos (angle p))))
  
  (defmethod cart-y ((p polar))
    (* (radius p) (sin (angle p))))
  
  (defgeneric to-cart (arg)
    (:method ((c cart))
    c)
  (:method ((p polar))
    (make-instance 'cart :x (cart-x p) :y (cart-y p))))
  
  (defmethod on-line-p ((c cart) x1 y1 x2 y2)
    (let ((a1 (- x1 x2)) (a2 (- y1 y2)))
      (cond ((< (abs a1) *tolerance*) 
                (< (abs (- (x c) x1)) *tolerance*))
            ((< (abs a2) *tolerance*) 
                (< (abs (- (y c) y1)) *tolerance*))
            (T (< (abs (- (/ (- (x c) x1) a1) 
                          (/ (- (y c) y1) a2))) 
                  *tolerance*)))))
  
  (defun on-single-line-p (vertices)
    (let ((x1 (x (to-cart (first vertices)))) 
          (y1 (y (to-cart (first vertices))))
          (x2 (x (to-cart (second vertices))))
          (y2 (y (to-cart (second vertices)))))
      (loop for point in vertices
        do (if (not (on-line-p (to-cart point) x1 y1 x2 y2))
          (return-from on-single-line-p NIL)))
    T))
\end{lstlisting}

\subsection{Результаты работы}
\begin{lstlisting}
dandachok@dpc:~/Documents/Study/sem6/FP/lab5$ clisp
[1]> (load "solution.lisp")
;; Loading file solution.lisp ...
;; Loaded file solution.lisp
#P"/home/dandachok/Documents/Study/sem6/FP/lab4/solution.lisp"
[38]> (setq c1 (make-instance 'cart :x 1 :y 1))
#<CART #x55E960AC1C68>
[38]> (setq c2 (make-instance 'cart :x 2 :y 2))
#<CART #x55E960ACF4F8>
[38]> (setq p1 (make-instance 'polar :radius 5 :angle (/ pi 4)))
#<POLAR #x55E960ADCE08>
[38]> (setq p2 (make-instance 'polar :radius 5 :angle (/ pi 6)))
#<POLAR #x55E960AEA718>
[38]> (setq l1 (list c1 c2 p1))
(#<CART #x55E960AC1C68> #<CART #x55E960ACF4F8> #<POLAR #x55E960ADCE08>)
[38]> (setq l2 (list c1 c2 p2))
(#<CART #x55E960AC1C68> #<CART #x55E960ACF4F8> #<POLAR #x55E960AEA718>)
[38]> (on-single-line-p l1)
T
[38]> (on-single-line-p l2)
NIL
\end{lstlisting}

\section{Дневник отладки}
\begin{tabular}{| l | m{3cm} | l | l |}
\hline
Дата & Событие & Действие по исправлению & Примечание \\
\hline
19.05.20 & Неправильное объявление функции & Заменить макрос {\tt defun} на {\tt defmethod} & \\
\hline
\end{tabular}

\section{Замечания автора по существу работы}

\begin{lstlisting}[language=lisp]

\end{lstlisting}

\section{Выводы}
Сложность программы по времени и по памяти линейная.
\end{document}