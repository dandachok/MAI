\documentclass{article}
\usepackage{amsmath,amsthm,amssymb}
\usepackage{mathtext}
\usepackage[T1,T2A, T2B]{fontenc}
\usepackage[utf8]{inputenc}
\usepackage[english,russian]{babel}
\usepackage{indentfirst}

\usepackage{graphicx}
\usepackage{pgfplots}

\renewcommand{\theenumi}{\arabic{enumi}}
\renewcommand{\labelenumi}{\arabic{enumi}}

\renewcommand{\theenumii}{\arabic{enumii}}
\renewcommand{\labelenumii}{\arabic{enumii})}

\oddsidemargin=-0.4mm
\textwidth=160mm
\topmargin=4.6mm
\textheight=210mm

\parindent=0pt
\parskip=3pt

\title{Домашняя работа}
\date{}

\pgfplotsset{compat=1.9}
\begin{document}

\begin{titlepage}

\vspace{100pt}
\begin{center}
    \huge \textbf{Домашняя работа}

    \vspace{50pt}
    
    \huge \textbf{Линейное программирование}
    
    \vspace{30pt}
    
    \huge \textbf{Симплекс-метод}
\end{center}
\begin{flushright}

\vspace{350pt}
\begin{tabular}{rl}
     \Large Студент: & \Large Д.Д.Наумов \\
     \Large Группа: & \Large 8О-306Б-17 \\
\end{tabular}
\end{flushright}
\end{titlepage}

\setcounter{section}{0}
\subsection*{Задание}

Фабрика производит 3 вида продукции, каждый из которых проходит обработку на токарном, фрезерном и сверлильном станках.

Затраты времени на обработку единицы продукции $j$-того типа на станке $i$-того типа составляют $a_{ij}$ единиц. 

Количество времени, которое может затратить станок $i$-того типа в неделю, ограничено и составляет $b_i$  единиц.

Прибыль от продажи единицы продукции $j$-того типа составляет $c_j$ единиц. 

Определить количество продукции каждого типа, которое должна произвести фабрика в течение недели из условия получения максимальной прибыли.
\subsection*{Вариант}
Номер по списку: 15

\begin{tabular}{| l | l | l | l | l | l | l |}
    \hline
    $$ & $1$ & $2$ & $3$ & $b_i$ & $c_i$ \\
    \hline
    $1$ & $10$ & $15$ & $5$ & $105000$ & $150$\\
    \hline
    $2$ & $20$ & $5$ & $5$ & $135000$ & $1500$\\
    \hline
    $3$ & $5$ & $10$ & $20$ & $180000$ & $7500$\\
    \hline
\end{tabular}

\pagebreak

\section*{Решение}

\begin{enumerate}
\item Математическая модель
    
    $\begin{aligned}
        &10x_1 + 15x_2 + 5x_3 \leqslant 105000 \\
        &20x_1 + 5x_2 + 5x_3 \leqslant 135000 \\
        &5x_1 + 10x_2 + 20x_3 \leqslant 180000 \\
        &F = 150x_1 + 1500x_2 + 7500x_3 \rightarrow max\\
    \end{aligned} \Longrightarrow \ \ 
    \begin{aligned}
        &10x_1 + 15x_2 + 5x_3 = 105000 \\
        &20x_1 + 5x_2 + 5x_3 = 135000 \\
        &5x_1 + 10x_2 + 20x_3 = 180000 \\
        &f = -F = -150x_1 - 1500x_2 - 7500x_3 \rightarrow min\\
    \end{aligned}$
\item Выражаем переменные
    
    $\begin{aligned}
        &x_4 = 105000 - 10x_1 - 15x_2 - 5x_3 \\
        &x_5 = 135000 - 20x_1 - 5x_2 - 5x_3 \\
        &x_6 = 180000 - 5x_1 - 10x_2 - 20x_3 \\
    \end{aligned}$
\item Cостовляем симплекс-таблицу

    \begin{tabular}{ l | l | l | l | l }
        & & $x_1$ & $x_2$ & $x_3$ \\
        \hline
        $x_4$ & $105000$ & $-10$ & $-15$ & $-5$ \\
        $x_5$ & $135000$ & $-20$ & $-5$ & $-5$ \\
        $x_6$ & $180000$ & $-5$ & $-10$ & $\boxed{-20}$ \\
        \hline
        $f$   & $0$ & $-150$ & $-1500$ & $-7500$ \\
    \end{tabular}

\item Работа с симлекс-таблицей

\begin{enumerate}
    \item все $\gamma_j$ отрицательные
    \item $\gamma_s = -7500$
    \item над $\gamma_s$ все $d_{is}$ отрицательные
    \item $\min\left\{\begin{aligned}
        \frac{105}{|-5|}, \frac{135}{|-5|}, \frac{180}{|-20|} 
    \end{aligned}\right\} = \begin{aligned}
        \frac{180}{|-20|}
    \end{aligned} \longrightarrow (-20)$ - ведущий элемент 
\end{enumerate}
\item Пересчет симплекс-таблицы

\renewcommand{\labelenumii}{\Roman{enumii}}

\begin{enumerate}
    \item Ведущий элемент
        
        $-20 \rightarrow -\frac{1}{20} = -0{,}05$
    \item Столбец ведущего элемента 
    
    $\begin{aligned}
        &-5 \rightarrow \frac{1}{4} = 0{,}25\\
        &-5 \rightarrow \frac{1}{4} = 0{,}25\\
        &-7500 \rightarrow 375 \\
    \end{aligned}$

    \item Строка ведущего элемента
    
    $\begin{aligned}
        &-10 \rightarrow -\frac{1}{4} = -0{,}25\\
        &-5 \rightarrow -\frac{1}{2} = -0{,}5\\
        &180000 \rightarrow 9000 \\
    \end{aligned}$

    \pagebreak

    \item Остальные элементы\\
    
    $\begin{aligned}
        &\begin{pmatrix} 105000 & -5 \\ 180000 & \boxed{-20} \end{pmatrix} \rightarrow 60000 \ & \
        &\begin{pmatrix} -10 & -5 \\ -5 & \boxed{-20} \end{pmatrix} \rightarrow -8{,}75 \ & \
        &\begin{pmatrix} -15 & -5 \\ -10 & \boxed{-20} \end{pmatrix} \rightarrow -12{,}5 \\
        &\begin{pmatrix} 135000 & -5 \\ 180000 & \boxed{-20} \end{pmatrix} \rightarrow 90000 \ & \
        &\begin{pmatrix} -10 & -5 \\ -5 & \boxed{-20} \end{pmatrix} \rightarrow -18{,}75 \ & \
        &\begin{pmatrix} -5 & -5 \\ -10 & \boxed{-20} \end{pmatrix} \rightarrow -2{,}5 \\
        &\begin{pmatrix} 180000 & \boxed{-20} \\ 0 & -7500 \end{pmatrix} \rightarrow -67500000 \ & \
        &\begin{pmatrix} -5 & \boxed{-20} \\ -150 & -7500 \end{pmatrix} \rightarrow 1725 \ & \
        &\begin{pmatrix} -10 & \boxed{-20} \\ -1500 & -7500 \end{pmatrix} \rightarrow 2250 \\
        \end{aligned}$
\end{enumerate}

\item Переходим к 3

\setcounter{enumi}{2}

\item Вторая симплекс-таблица

\begin{tabular}{ l | l | l | l | l }
    & & $x_1$ & $x_2$ & $x_6$ \\
    \hline
    $x_4$ & $60000$ & $-8{,}75$ & $-12{,}5$ & $-0{,}25$ \\
    $x_5$ & $90000$ & $-18{,}75$ & $-2{,}5$ & $-0{,}25$ \\
    $x_3$ & $9000$ & $-0{,}25$ & $-0{,}5$ & $-0{,}05$ \\
    \hline
    $f$   & $-67500000$ & $1725$ & $2250$ & $375$ \\
\end{tabular}

\item Проверка условия

    Все $\gamma_j > 0 \Rightarrow$ симплекс-таблица оптимальна
\end{enumerate}

Ответ: $x_1 = 0,\ x_2 = 0,\ x_3 = 9000$.
\end{document}
