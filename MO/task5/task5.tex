\documentclass{article}
\usepackage{amsmath,amsthm,amssymb}
\usepackage{mathtext}
\usepackage[T1,T2A, T2B]{fontenc}
\usepackage[utf8]{inputenc}
\usepackage[english,russian]{babel}
\usepackage{indentfirst}


\usepackage{graphicx}
\usepackage{pgfplots}

\renewcommand{\thesubsection}{\Roman{subsection} Фаза}
\renewcommand{\theenumi}{\arabic{enumi}}
\renewcommand{\labelenumi}{\arabic{enumi}}
\renewcommand{\theenumii}{\arabic{enumii}}
\renewcommand{\labelenumii}{\arabic{enumii})}

\oddsidemargin=-0.4mm
\textwidth=160mm
\topmargin=4.6mm
\textheight=210mm

\parindent=0pt
\parskip=3pt

\title{Домашняя работа}
\date{}

\pgfplotsset{compat=1.9}
\begin{document}

\begin{titlepage}

\vspace{100pt}
\begin{center}
    \huge \textbf{Домашняя работа}

    \vspace{50pt}
    
    \huge \textbf{Линейное программирование}
    
    \vspace{30pt}
    
    \huge \textbf{Двухфазный симплекс-метод}
\end{center}
\begin{flushright}

\vspace{350pt}
\begin{tabular}{rl}
     \Large Студент: & \Large Д.Д.Наумов \\
     \Large Группа: & \Large 8О-306Б-17 \\
\end{tabular}
\end{flushright}
\end{titlepage}

\setcounter{section}{0}
\subsection*{Задание}
Человек должен потреблять в сутки определенное количество питательных веществ $A$, $B$, $C$.
Количества этих веществ в различных видах пищи $P_i$ даны в таблице. Здесь же указаны и цены единицы пищи.

\subsection*{Вариант}
Номер по списку: 15

\begin{tabular}{| l | l | l | l |}
    \hline
    Питательные вещества & Норма & $P_1$ & $P_2$ \\
    \hline
    $A$ & $150$ & $5$ & $2$ \\
    \hline
    $B$ & $90$ & $1$ & $2$ \\
    \hline
    $C$ & $450$ & $10$ & $6$ \\
    \hline
    Цена & & $10$ & $6$ \\
    \hline
\end{tabular}

\pagebreak

\section*{Решение}
\subsection{}
\begin{enumerate}
\item Математическая модель
    
    $\begin{aligned}
        &5x_1 + 2x_2 \geqslant 150 \\
        &x_1 + 2x_2 \geqslant 90 \\
        &10x_1 + 6x_2 \geqslant 450 \\
        &f = 10x_1 + 6x_2 \rightarrow min\\
    \end{aligned} \Longrightarrow \ \ 
    \begin{aligned}
        &5x_1 + 2x_2 - x_3 = 150 \\
        &x_1 + 2x_2 - x_4 = 90 \\
        &10x_1 + 6x_2 - x_5 = 450 \\
        &f = 10x_1 + 6x_2 \rightarrow min\\
    \end{aligned}$
\item Составляем вспомогательную ЗЛП
    
    $\begin{aligned}
        &\xi_1 = 150 - 5x_1 - 2x_2 + x_3 \\
        &\xi_2 = 90 - x_1 - 2x_2 + x_4 \\
        &\xi_3 = 450 - 10x_1 - 6x_2 + x_5 \\
        &\Phi = 690 
        - 16x_1 - 10x_2 + x_3 + x_4 + x_5 \rightarrow min\\
    \end{aligned}$
\item Решение ВЗЛП
\begin{center}
        
    \begin{tabular}{ c | c | c | c | c | c | c}
        & $\beta$ & $x_1$ & $x_2$ & $x_3$ & $x_4$ & $x_5$\\
        \hline
        $\xi_1$ & $150$ & $\boxed{-5}$ & $-2$ & $1$  & $0$ & $0$ \\
        $\xi_2$ & $90$ & $-1$ & $-2$ & $0$ & $1$ & $0$ \\
        $\xi_3$ & $450$ & $-10$ & $-6$ & $0$ & $0$ & $1$ \\
        \hline
        $\Phi$   & $690$ & $-16$ & $-10$ & $1$ & $1$ & $1$ \\
    \end{tabular}

    $\Downarrow$

    \begin{tabular}{ c | c | c | c | c | c | c}
        & $\beta$ & $\xi_1$ & $x_2$ & $x_3$ & $x_4$ & $x_5$\\
        \hline
        $x_1$ & $30$ & $-0{,}2$ & $-0{,}4$ & $0{,}2$  & $0$ & $0$ \\
        $\xi_2$ & $60$ & $0{,}2$ & $\boxed{-1{,}6}$ & $-0{,}2$ & $1$ & $0$ \\
        $\xi_3$ & $150$ & $2$ & $-2$ & $-2$ & $0$ & $1$ \\
        \hline
        $\Phi$   & $210$ & $3{,}2$ & $-3{,}6$ & $-2{,}2$ & $1$ & $1$ \\
    \end{tabular}

    $\Downarrow$

    \begin{tabular}{ c | c | c | c | c | c | c}
        & $\beta$ & $\xi_1$ & $\xi_2$ & $x_3$ & $x_4$ & $x_5$\\
        \hline
        $x_1$   & $15$ & $-0{,}25$ & $0{,}25$ & $0{,}25$ & $-0{,}25$ & $0$ \\
        $x_2$ & $37{,}5$ & $0{,}125$ & $-0{,}625$ & $-0{,}125$ & $0.625$ & $0$ \\
        $\xi_3$ & $75$ & $1{,}75$ & $1{,}25$ & $\boxed{-1{,}75}$ & $-1.25$ & $1$ \\
        \hline
        $\Phi$   & $75$ & $2{,}75$ & $2{,}25$ & $-1{,}75$ & $-1{,}25$ & $1$ \\
    \end{tabular}

    $\Downarrow$

    \begin{tabular}{ c | c | c | c | c | c | c}
        & $\beta$ & $\xi_1$ & $\xi_2$ & $\xi_3$ & $x_4$ & $x_5$\\
        \hline
        & & & & & & \\
        $x_1$ & $\begin{aligned}\frac{180}{7}\end{aligned}$ & $0$ & $\begin{aligned}\frac{3}{7}\end{aligned}$ & $-\begin{aligned}\frac{1}{7}\end{aligned}$ & $-\begin{aligned}\frac{3}{7}\end{aligned}$ & $\begin{aligned}\frac{1}{7}\end{aligned}$ \\
        & & & & & & \\
        $x_2$ & $\begin{aligned}\frac{225}{7}\end{aligned}$ & $0$ & $-\begin{aligned}\frac{5}{7}\end{aligned}$ & $\begin{aligned}\frac{1}{14}\end{aligned}$ & $\begin{aligned}\frac{5}{7}\end{aligned}$ & $-\begin{aligned}\frac{1}{14}\end{aligned}$ \\
        & & & & & & \\
        $x_3$ & $\begin{aligned}\frac{300}{7}\end{aligned}$ & $1$ & $\begin{aligned}\frac{5}{7}\end{aligned}$  & $-\begin{aligned}\frac{4}{7}\end{aligned}$ & $-\begin{aligned}\frac{5}{7}\end{aligned}$ & $\begin{aligned}\frac{4}{7}\end{aligned}$ \\
        & & & & & & \\
        \hline
        & & & & & & \\
        $\Phi$   & $0$ & $1$ & $1$ & $1$ & $0$ & $0$ \\
    \end{tabular}
\end{center}
\item $\min\Phi \ngtr 0$
\item Все $\xi_i$ свободные
\item Симплекс-таблица для исходной ЗЛП

\begin{tabular}{ c | c | c | c}
    & $\beta$ & $x_4$ & $x_5$\\
    \hline
    & & & \\
    $x_1$ & $\begin{aligned}\frac{180}{7}\end{aligned}$ & $-\begin{aligned}\frac{3}{7}\end{aligned}$ & $\begin{aligned}\frac{1}{7}\end{aligned}$ \\
    & & &\\
    $x_2$ & $\begin{aligned}\frac{225}{7}\end{aligned}$ & $\begin{aligned}\frac{5}{7}\end{aligned}$ & $-\begin{aligned}\frac{1}{14}\end{aligned}$ \\
    & & & \\
    $x_3$ & $\begin{aligned}\frac{300}{7}\end{aligned}$ & $-\begin{aligned}\frac{5}{7}\end{aligned}$ & $\begin{aligned}\frac{4}{7}\end{aligned}$ \\
    & & & \\
    \hline
    & & & \\
    $f$ & $450$ & $0$ & $1$ \\
\end{tabular}

\item все $\gamma_i \geqslant 0 \Rightarrow $ таблица оптимальна
 
\end{enumerate}

Ответ: $x_1 = \begin{aligned}\frac{180}{7},\ x_2 = \frac{225}{7}\end{aligned}$.

\usepgflibrary[patterns]
\usetikzlibrary{patterns}.

\begin{tikzpicture}
    \begin{axis} [
        scale = 2,
        xmin = 0,
        ymin = 0,
        xmax = 100,
        ymax = 100,
        samples = 100,
    ]
    \addplot[blue, pattern = crosshatch dots, pattern color = blue, domain=1:90]{30 - 0.4*x} \closedcycle;
    \addplot[gray, pattern = {north east lines}, pattern color = gray, domain=1:90]{90 - 2*x} \closedcycle;
    \addplot[red, pattern = crosshatch dots, pattern color = red,domain=1:90]{45 - 0.6*x} \closedcycle;
    \end{axis}
\end{tikzpicture}

\end{document}
