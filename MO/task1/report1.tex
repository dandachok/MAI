\documentclass{article}
\usepackage{amsmath,amsthm,amssymb}
\usepackage{mathtext}
\usepackage[T1,T2A, T2B]{fontenc}
\usepackage[utf8]{inputenc}
\usepackage[english,russian]{babel}
\usepackage{indentfirst}

\usepackage{graphicx}
\usepackage{pgfplots}

\oddsidemargin=-0.4mm
\textwidth=160mm
\topmargin=4.6mm
\textheight=210mm

\parindent=0pt
\parskip=3pt

\title{Расчетно-графическая работа}
\date{}

\pgfplotsset{compat=1.9}
\begin{document}

\begin{titlepage}

\vspace{100pt}
\begin{center}
    \huge \textbf{Расчетно-графическая работа} \\
    \vspace{50pt}
    \huge \textbf{Методы безусловной оптимизации}
\end{center}
\begin{flushright}

\vspace{350pt}
\begin{tabular}{rl}
     \Large Студент: & \Large Д.Д.Наумов \\
     \Large Группа: & \Large 8О-306Б-17 \\
\end{tabular}
\end{flushright}
\end{titlepage}

\setcounter{section}{0}

\section{Классический метод}

\textbf{Функция}: $f(x) = 2x_1^2 + x_1x_2 + 3x_2^2 + 45x_1 - 15x_2$

\textbf{Задание}: Найти экстремум и определить его тип ($max$ или $min$) для заданной функции $f(x)$ классическим методом, используя  необходимые и достаточные условия существования экстремума.

\textbf{Решение:}

\begin{enumerate}
    \setcounter{enumi}{0}
    \item 
        $\frac{\partial f(x)}{\partial x_1} = 4x_1 + x_2 + 45$
    
        $\frac{\partial f(x)}{\partial x_2} = x_1 + 6x_2 - 15$
    \item
        $\begin{cases}
            4x_1 + x_2 + 45 = 0 \\
            x_1 + 6x_2 - 15 = 0
        \end{cases}
        \Longrightarrow \ \ 
        \begin{cases}
            -23x_2 = -105 \\
            x_1 = 15 - 6x_2
        \end{cases}
        \Longrightarrow
        \begin{cases}
            x_2 = \frac{105}{23} \approx 4.6 \\
            x_1 = - \frac{285}{23} \approx -12.4
        \end{cases}$ 
        
        $x^{ст} = \begin{pmatrix}
             -12.4  \\
             4.6 
        \end{pmatrix}$
    \item
        $\frac{\partial^2 f(x)}{\partial^2 x_1} = 4$ \ \ 
        $\frac{\partial^2 f(x)}{\partial^2 x_2} = 6$ \ \ 
        $\frac{\partial^2 f(x)}{\partial x_1 \partial x_2} = 1$
        
        $\Gamma = 
        \begin{pmatrix}
            4& 1 \\
            1& 6
        \end{pmatrix}
        $
    \item
        $\Delta_1 = 4$
        
        $\Delta_2 = 
        \begin{vmatrix}
            4 & 1 \\
            1 & 6 \\
        \end{vmatrix} = 23$
    \item
        $\Delta_1 > 0, \Delta_2 > 0 \Rightarrow x^{ст} - \text{точка } min$
\end{enumerate}

\pagebreak

\section{Градиентный метод с постоянным шагом}

\textbf{Задание}: Задать начальную точку и выполнить четыре шага градиентным методом с постоянным шагом.

\textbf{Решение}:

\begin{enumerate}
    \item
        $f(x) = 2x_1^2 + x_1x_2 + 3x_2^2 + 45x_1 - 15x_2$
        
        $x^{(0)} = \begin{pmatrix} 0 \\ 0 \end{pmatrix}$, 
        $h = 0.1, \ \varepsilon = 0.1, \ k= 0$
        
        $\nabla f(x) =
        \begin{pmatrix}
            4x_1 + x_2 + 45 \\
            x_1 + 6x_2 - 15
        \end{pmatrix}$
\end{enumerate}

\ \ \ \ Итерация 1:

\begin{enumerate}
    \setcounter{enumi}{1}
    \item $\nabla f(x^{(0)}) = \begin{pmatrix} 45 \\ -15 \end{pmatrix}$
    \item $|\nabla f(x^{(0)})| \approx 47 > \varepsilon$
    
        Условие окончания не выполнено
    \item $x^{(1)} = x^{(0)} - h\nabla f(x^{(0)}) =
        \begin{pmatrix} 0 \\ 0 \end{pmatrix} - 0.1
        \begin{pmatrix} 45 \\ -15 \end{pmatrix} =
        \begin{pmatrix} -4.5 \\ 1.5 \end{pmatrix}$
    \item $f(x^{(1)}) = -184.5 < f(x^{(0)}) = 0$
    \item $k = 1$
\end{enumerate}

\ \ \ \ Итерация 2:

\begin{enumerate}
    \setcounter{enumi}{1}
    \item $\nabla f(x^{(1)}) = \begin{pmatrix} 28.5 \\ -10.5 \end{pmatrix}$
    \item $|\nabla f(x^{(1)})| \approx 30 > \varepsilon$
    
        Условие окончания не выполнено
    \item $x^{(2)} = x^{(1)} - h\nabla f(x^{(1)}) =
        \begin{pmatrix} -4.5 \\ 1.5 \end{pmatrix} - 0.1
        \begin{pmatrix} 28.5 \\ -10.5 \end{pmatrix} =
        \begin{pmatrix} -7.35 \\ 2.55 \end{pmatrix}$
    \item $f(x^{(2)}) = -260 < f(x^{(1)}) = -184$
    \item $k = 2$
\end{enumerate}

\ \ \ \ Итерация 3:

\begin{enumerate}
    \setcounter{enumi}{1}
    \item $\nabla f(x^{(2)}) = \begin{pmatrix} 18.15 \\ -7.05 \end{pmatrix}$
    \item $|\nabla f(x^{(2)})| \approx 19.5 > \varepsilon$
    
        Условие окончания не выполнено
    \item $x^{(3)} = x^{(2)} - h\nabla f(x^{(2)}) =
        \begin{pmatrix} -7.35 \\ 2.55 \end{pmatrix} - 0.1
        \begin{pmatrix} 18.15 \\ -7.05 \end{pmatrix} =
        \begin{pmatrix} -9.165 \\ 3.255 \end{pmatrix}$
    \item $f(x^{(3)}) = -291 < f(x^{(2)}) = -260$
    \item $k = 3$
\end{enumerate}

\ \ \ \ Итерация 4:

\begin{enumerate}
    \setcounter{enumi}{1}
    \item $\nabla f(x^{(3)}) = \begin{pmatrix} 11.595 \\ -4.635 \end{pmatrix}$
    \item $|\nabla f(x^{(3)})| \approx 12.5 > \varepsilon$
    
        Условие окончания не выполнено
    \item $x^{(4)} = x^{(3)} - h\nabla f(x^{(3)}) =
        \begin{pmatrix} -9.165 \\ 3.255 \end{pmatrix} - 0.1
        \begin{pmatrix} 11.595 \\ -4.635 \end{pmatrix} =
        \begin{pmatrix} -10.3245 \\ 3.7 \end{pmatrix}$
    \item $f(x^{(4)}) = -304 < f(x^{(3)}) = -291$
    \item $k = 4$
\end{enumerate}

\pagebreak

\section{Метод наискорейшего спуска}

\textbf{Задание}: Задать начальную точку и выполнить три шага методом наискорейшего спуска. 

\textbf{Решение}:

\begin{enumerate}
    \item
        $f(x) = 2x_1^2 + x_1x_2 + 3x_2^2 + 45x_1 - 15x_2$
        
        $x^{(0)} = \begin{pmatrix} 0 \\ 0 \end{pmatrix}$, 
        $h = 0.1, \ \varepsilon = 0.1, \ k= 0$
        
        $\nabla f(x) =
        \begin{pmatrix}
            4x_1 + x_2 + 45 \\
            x_1 + 6x_2 - 15
        \end{pmatrix}$
\end{enumerate}

\ \ \ \ Итерация 1:

\begin{enumerate}
    \setcounter{enumi}{1}
    \item $\nabla f(x^{(0)}) = \begin{pmatrix} 45 \\ -15 \end{pmatrix}$
    \item $|\nabla f(x^{(0)})| \approx 47 > \varepsilon$
    
        Условие окончания не выполнено
    \item $x^{(1)}(h_{0}) = x^{(0)} - h_{0}\nabla f(x^{(0)}) =
        \begin{pmatrix} 0 \\ 0 \end{pmatrix} - h_{0}
        \begin{pmatrix} 45 \\ -15 \end{pmatrix} =
        \begin{pmatrix} -4.5h_{0} \\ 1.5h_{0} \end{pmatrix}$
        
        
        $\begin{pmatrix}
            \begin{pmatrix} 45 \\ -15 \end{pmatrix},
            \begin{pmatrix} -165h_{0} + 45 \\ 105h_{0} - 15 \end{pmatrix}
        \end{pmatrix} = -9000h_{0} + 2250 = 0$
        
        
        $h_{0} = 0.25$
    \item $x^{(1)} = \begin{pmatrix} -11.25 \\ 3.75 \end{pmatrix}$
    \item $k = 1$
\end{enumerate}

\ \ \ \ Итерация 2:

\begin{enumerate}
    \setcounter{enumi}{1}
    \item $\nabla f(x^{(1)}) = \begin{pmatrix} 3.75 \\ -3.75 \end{pmatrix}$
    \item $|\nabla f(x^{(1)})| \approx 5.3 > \varepsilon$
    
        Условие окончания не выполнено
    \item $x^{(2)}(h_{1}) = x^{(1)} - h_{1}\nabla f(x^{(1)}) =
        \begin{pmatrix} -11.25 \\ 3.75 \end{pmatrix} - h_{1}
        \begin{pmatrix} 3.75 \\ -3.75 \end{pmatrix} =
        \begin{pmatrix} -11.25 - 3.75h_{1} \\ 3.75 + 3.75h_{1} \end{pmatrix}$
        
        
        $\begin{pmatrix}
            \begin{pmatrix} 3.75 \\ -3.75 \end{pmatrix},
            \begin{pmatrix} 3.75 - 11.25h_{1} \\ -3.75 + 18.75h_{1} \end{pmatrix}
        \end{pmatrix} = 22.7138 - 112.5h_{1} = 0$
        
        
        $h_{1} = 0.20$
    \item $x^{(2)} = \begin{pmatrix} -12 \\ 4.5 \end{pmatrix}$
    \item $k = 2$
\end{enumerate}

\ \ \ \ Итерация 3:

\begin{enumerate}
    \setcounter{enumi}{1}
    \item $\nabla f(x^{(2)}) = \begin{pmatrix} 1.5 \\ 0 \end{pmatrix}$
    \item $|\nabla f(x^{(2)})| = 1.5 > \varepsilon = 0.1$
    
        Условие окончания не выполнено
    \item $x^{(3)}(h_{2}) = x^{(2)} - h_{2}\nabla f(x^{(2)}) =
        \begin{pmatrix} -12 \\ 4.5 \end{pmatrix} - h_{2}
        \begin{pmatrix} 1.5 \\ 0 \end{pmatrix} =
        \begin{pmatrix} -12 - 1.5h_{2} \\ 4.5 \end{pmatrix}$
        
        
        $\begin{pmatrix}
            \begin{pmatrix} 1.5 \\ 0 \end{pmatrix},
            \begin{pmatrix} 1.5 - 6h_{2} \\ -1.5h_{2} \end{pmatrix}
        \end{pmatrix} = 2.25 - 9h_{2} = 0$
        
        
        $h_{2} = 0.25$
    \item $x^{(3)} = \begin{pmatrix} -12.375 \\ 4.5 \end{pmatrix}$
    \item $k = 3$
\end{enumerate}

\pagebreak

\section{Метод покоординатного спуска}

\textbf{Задание}: Задать начальную точку и выполнить три шага методом покоординатного спуска.

\textbf{Решение}:

\begin{enumerate}
    \item
        $f(x) = 2x_1^2 + x_1x_2 + 3x_2^2 + 45x_1 - 15x_2$
        
        $x^{(0)} = \begin{pmatrix} 0 \\ 0 \end{pmatrix}$, 
        $h = 0.1, \ \varepsilon = 0.1, \ n = 2$
        
        $\nabla f(x) =
        \begin{pmatrix}
            4x_1 + x_2 + 45 \\
            x_1 + 6x_2 - 15
        \end{pmatrix}$
    \item $l = 0$
    \item $f(x^{(0)}) = 0$
\end{enumerate}

\ \ \ \ Итерация 1:

\begin{enumerate}
    \setcounter{enumi}{3}
    \item $k = 1$
    \item $\frac{\partial f(x^{k+nl-1})}{\partial x_k} = \frac{\partial f(x^{(0)})}{\partial x_1} = 45$
    \item $\left|\left|\frac{\partial f(x^{(0)})}{\partial x_1}\right|\right| < \varepsilon$ - нет
    \item $e_{1} = \begin{pmatrix} 1 \\ 0 \end{pmatrix}$
    \item $x^{(1)} = x^{(0)} - h \frac{\partial f(x^{(0)})}{\partial x_1} e_1 = 
        \begin{pmatrix} 0 \\ 0 \end{pmatrix} - 0.1 \cdot (45) \cdot \begin{pmatrix} 1 \\ 0 \end{pmatrix} = 
        \begin{pmatrix} -4.5 \\ 0 \end{pmatrix}$
    \item $f(x^{(1)}) = -162.0$
    \item $f(x^{(1)}) = -162 < f(x^{(0)}) = 0$
    \item $k \ne n$
    \item k = 2
\end{enumerate}

\ \ \ \ Итерация 2:

\begin{enumerate}
    \setcounter{enumi}{4}
    \item $\frac{\partial f(x^{(1)})}{\partial x_2} = -19.5$
    \item $\left|\left|\frac{\partial f(x^{(1)})}{\partial x_2}\right|\right| > \varepsilon$
    \item $e_{2} = \begin{pmatrix} 0 \\ 1 \end{pmatrix}$
    \item $x^{(2)} = x^{(1)} - h \frac{\partial f(x^{(1)})}{\partial x_2} e_2 = 
        \begin{pmatrix} -4.5 \\ 0 \end{pmatrix} - 0.1 \cdot (-19.5) \cdot \begin{pmatrix} 0 \\ 1 \end{pmatrix} = 
        \begin{pmatrix} -4.5 \\ 1.95 \end{pmatrix}$
    \item $f(x^{(2)}) = -188$
    \item $f(x^{(2)}) = -188 < f(x^{(1)}) = -162$
    \item $k = n \ \ \Rightarrow \ \ l = 1$
\end{enumerate}

\ \ \ \ Итерация 3:

\begin{enumerate}
    \setcounter{enumi}{3}
    \item $k = 1$
    \item $\frac{\partial f(x^{(2)})}{\partial x_1} = 28.95$
    \item $\left|\left|\frac{\partial f(x^{(2)})}{\partial x_1}\right|\right| > \varepsilon$
    \item $e_{1} = \begin{pmatrix} 1 \\ 0 \end{pmatrix}$
    \item $x^{(3)} = x^{(2)} - h \frac{\partial f(x^{(2)})}{\partial x_1} e_1 = 
        \begin{pmatrix} -4.5 \\ 1.95 \end{pmatrix} - 0.1 \cdot (28.95) \cdot \begin{pmatrix} 1 \\ 0 \end{pmatrix} = 
        \begin{pmatrix} -7.395 \\ 1.95 \end{pmatrix}$
    \item $f(x^{(3)}) = -255.67$
    \item $f(x^{(3)}) = -255 < f(x^{(2)}) = -188$
    \item $k \ne n$
    \item k = 2
\end{enumerate}

\pagebreak

\section{Метод Гаусса-Зейделя}

\textbf{Задание}: Задать начальную точку и выполнить два шага методом Гаусса-Зейделя.

\textbf{Решение}:

\begin{enumerate}
    \item
        $f(x) = 2x_1^2 + x_1x_2 + 3x_2^2 + 45x_1 - 15x_2$
        
        $x^{(0)} = \begin{pmatrix} 0 \\ 0 \end{pmatrix}$, 
        $\varepsilon = 0.1, \ n = 2$
        
    \item $l = 0$
    \item $f(x^{(0)}) = 0$
\end{enumerate}

\ \ \ \ Итерация 1:

\begin{enumerate}
    \setcounter{enumi}{3}
    \item $k = 1$
    \item $\frac{\partial f(x^{k+nl-1})}{\partial x_k} = \frac{\partial f(x^{(0)})}{\partial x_1} = 45$
    \item $\left|\left|\frac{\partial f(x^{(0)})}{\partial x_1}\right|\right| > \varepsilon$ - продолжаем
    \item $e_{1} = \begin{pmatrix} 1 \\ 0 \end{pmatrix}$
    \item $x^{(1)}(h_0) = x^{(0)} - h_0 \frac{\partial f(x^{(0)})}{\partial x_1} e_1 = 
        \begin{pmatrix} 0 \\ 0 \end{pmatrix} - h_0 \cdot (45) \cdot \begin{pmatrix} 1 \\ 0 \end{pmatrix} = 
        \begin{pmatrix} -45h_0 \\ 0 \end{pmatrix}$
        
    \item $f(x^{(1)}(h_0)) = 4050h_0^2 - 2025h_0$
    \item $\frac{\partial f(x^{(1)}(h_0))}{\partial h_0} = 8100h_0 - 2025 = 0$
        
        $h_0 = 0.25$
    \item $x^{(1)} = \begin{pmatrix} -11.25 \\ 0 \end{pmatrix}$
    \item $f(x^{(1)}) = -253 < f(x^{(0)}) = 0$
    \item $k \ne n \ \Rightarrow \ k = 2$
\end{enumerate}

\ \ \ \ Итерация 2:

\begin{enumerate}
    \setcounter{enumi}{4}
    \item $\frac{\partial f(x^{(1)})}{\partial x_2} = -26.25$
    \item $\left|\left|\frac{\partial f(x^{(1)})}{\partial x_2}\right|\right| > \varepsilon$ - продолжаем
    \item $e_{2} = \begin{pmatrix} 0 \\ 1 \end{pmatrix}$
    \item $x^{(2)}(h_1) = x^{(1)} - h_1 \frac{\partial f(x^{(1)})}{\partial x_2} e_2 = 
        \begin{pmatrix} -11.25 \\ 0 \end{pmatrix} - h_1 \cdot (-26.25) \cdot \begin{pmatrix} 0 \\ 1 \end{pmatrix} = 
        \begin{pmatrix} -11.25 \\ 26.25h_1 \end{pmatrix}$
        
    \item $f(x^{(2)}(h_1)) = 2067h_1^2 - 689.0625h_1 - 253$
    \item $\frac{\partial f(x^{(2)}(h_1))}{\partial h_1} = 4134h_1 - 689 = 0$
        
        $h_1 \approx 0.17$
    \item $x^{(2)} = \begin{pmatrix} -11.25 \\ 4.46 \end{pmatrix}$
    \item $f(x^{(2)}) = -310 < f(x^{(1)}) = -253$
    \item $k = n \ \Rightarrow \ l = 1$
    
    Переходим к 4
\end{enumerate}

\pagebreak

\section{Метод Ньютона}

\textbf{Задание}: Задать начальную точку и выполнить один шаг методом Ньютона. 

\textbf{Решение}:

\begin{enumerate}
    \item
        $f(x) = 2x_1^2 + x_1x_2 + 3x_2^2 + 45x_1 - 15x_2$
        
        $x^{(0)} = \begin{pmatrix} 0 \\ 0 \end{pmatrix}$, 
        $\varepsilon = 0.1$
        
        $\nabla f(x) =
        \begin{pmatrix}
            4x_1 + x_2 + 45 \\
            x_1 + 6x_2 - 15
        \end{pmatrix}$
    \item $\nabla f(x^{(0)}) =
    \begin{pmatrix}
        45 \\
        -15
    \end{pmatrix}$
    \item $\left| \nabla f(x^{(0)}) \right| = \sqrt{45^2 + (-15)^2} \approx 47 > \varepsilon$
    
    Критерий окончания не выполнено
    \item $\frac{\partial ^2 f(x)}{\partial x_1^2} = 4$ 
        $\frac{\partial ^2 f(x)}{\partial x_2^2} = 6$
        $\frac{\partial ^2 f(x)}{\partial x_1 \partial x_2} = 1$

        $\Gamma = \begin{pmatrix} 4 & 1 \\ 1 & 6 \end{pmatrix}$
    \item $\Gamma(x_0) = \begin{pmatrix} 4 & 1 \\ 1 & 6 \end{pmatrix}$
    \item $\Gamma^{-1}(x_0) = \begin{pmatrix} \frac{6}{23} & \frac{-1}{23} \\\\ \frac{-1}{23} & \frac{4}{23} \end{pmatrix}$
    \item $\Delta_1 = \frac{6}{23} > 0, \ $ 
        $\Delta_2 = \begin{vmatrix} \frac{6}{23} & \frac{-1}{23} \\\\ \frac{-1}{23} & \frac{4}{23} \end{vmatrix} = 1 > 0 \ \Rightarrow \ \Gamma^{-1}(x_0) > 0$
    \item $x^{(1)} = x^{(0)} - \Gamma^{-1}(x_0) \nabla f(x_0) = $
        $\begin{pmatrix} 0 \\ 0 \end{pmatrix} - $
        $\begin{pmatrix} \frac{6}{23} & \frac{-1}{23} \\\\ \frac{-1}{23} & \frac{4}{23} \end{pmatrix}$
        $\begin{pmatrix} 45 \\ -15 \end{pmatrix} = \begin{pmatrix} \frac{-275}{23} \\\\ \frac{105}{23} \end{pmatrix} \approx$
        $\begin{pmatrix} -12 \\ 4.6 \end{pmatrix}$

\end{enumerate}

\pagebreak

\section{Иллюстрация методов}

\begin{tikzpicture}
\begin{axis} [
    legend pos = {north east},
    width = 450,
    height = 450,
    xlabel = {$x1$},
    ylabel = {$x2$},
    axis y line = right,
    axis x line = left,
    xmin = -14,
    ymax = 5,
    grid=major,
]
\legend {
    классический,
    градиентный,
    метод наискорейшего спуска,
    метод покоординатного спуска,
    метод Гаусса-Зейделя,
    метод Ньютона
};
\addplot [ultra thick, blue, mark=*]coordinates {
    (0, 0) (-12.4, 4.6)
};
\addplot [ultra thick, red, mark=*]coordinates {
    (0, 0) (-4.5, 1.5) (-7.35, 2.55) (-9.165, 3.255) (-10.3245, 3.7)
};
\addplot [ultra thick, brown, mark=*]coordinates {
    (0, 0) (-11.25, 3.75) (-12, 4.5) (-12.375, 4.5)
};
\addplot [ultra thick, color=darkgray, mark=*] coordinates {
    (0, 0) (-4.5, 0) (-4.5, 1.95) (-7.395, 1.95)
};
\addplot [ultra thick, color=magenta, mark=*] coordinates {
    (0, 0) (-11.25, 0) (-11.25, 4.46)
};
\addplot [ultra thick, color=orange, mark=*] coordinates {
    (0, 0) (-12.0, 4.6)
};
\end{axis}
\end{tikzpicture}

\end{document}
